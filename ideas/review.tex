%\documentclass[iop]{emulateapj}
%\documentclass[12pt, preprint]{emulateapj}
\documentclass[12pt, onecolumn]{emulateapj}

\usepackage{amsmath}
%\usepackage{bibtex}
%\bibliographystyle{unsrtnat}

\usepackage{tikz}
\usetikzlibrary{shapes.geometric, arrows}
\usetikzlibrary{fit}

\tikzstyle{hyper} = [circle, text centered, draw=black]%, fill=blue!30]
\tikzstyle{param} = [circle, text centered, draw=black]%, fill=green!30]
\tikzstyle{data} = [circle, text centered, draw=black, line width=2pt]%, fill=red!30]
%\tikzstyle{hyper} = [trapezium, trapezium left angle=70, trapezium right angle=110, minimum width=1cm, minimum height=0.5cm, text centered, draw=black, fill=green!30]
%\tikzstyle{param} = [rectangle, minimum width=1cm, minimum height=0.5cm, text centered, draw=black, fill=green!30]
%\tikzstyle{data} = [diamond, minimum width=1cm, minimum height=1cm, text centered, draw=black, fill=red!30]
%\tikzstyle{eqn} = [rectangle, minimum width=1cm, minimum height=0.5cm, text centered, draw=black]%, fill=green!30]
%\tikzstyle{latent} = [diamond, minimum width=1cm, minimum height=0.5cm, text centered, draw=black]%, fill=green!30]
\tikzstyle{arrow} = [thick,->,>=stealth]

\newcommand{\myemail}{aimalz@nyu.edu}
\newcommand{\textul}{\underline}

%\slugcomment{}

\shorttitle{Cross-correlation Redshifts}
\shortauthors{Malz}

\begin{document}

\begin{align}
\end{align}

\title{Cross-correlation Redshift Estimation}

\author{A.I. Malz\altaffilmark{1}}
\altaffiltext{1}{CCPP}
\email{aimalz@nyu.edu}

\begin{abstract}
This document presents a review of cross-correlation redshift distribution estimation and extends the method to enable determination of redshift posteriors for individual galaxies.
\end{abstract}

\keywords{photo-z}

\section{Cross-correlation Redshift Density Estimation}
\label{sec:review}

This section outlines the method proposed in \citet{new08} for determining the redshift probability distribution $\phi_{P}(z)$ for a sample of $P$ galaxies $p$ with photometric redshifts $\hat{z}_{p}$ without spectroscopic confirmation given their angular positions $\vec{\theta}_{p}$ as well as a set of $S$ galaxies with spectroscopic redshifts $z_{s}$ and their angular positions $\vec{\theta}_{s}$.  We would like to take advantage of the observed angular cross-correlation function $w_{SP}(\theta,z)$ as a prior of sorts on the redshift probability distribution in question.

\begin{enumerate}

\item Calculate the mean surface density $\bar{\Sigma}_{P}=\frac{N_{P}}{\Omega_{P}(\{\vec{\theta}_{p}\}_{P})}$ of galaxies in $P$ in terms of the total number of galaxies $N_{P}$ in $P$ and the angular area $\Omega_{P}(\{\vec{\theta}_{p}\}_{P})$ spanned by the galaxies in $P$.

\item Calculate surface density $\Sigma_{sP}(\theta,z_{s})=\frac{N_{P_{s}}}{\Omega(\theta)}$ of the $N_{P_{s}}$ galaxies in $P$ a distance $\theta$ from each galaxy $s$ in $S$ at redshift $z_{s}$, where $\Omega(\theta)$ is the angular area defined by a circle of radius $\theta$ around the coordinates of galaxy $s$ .

\item Calculate the average of $\Sigma_{sP}(\theta,z_{s})$ over all $\mathcal{S}_{s}$ galaxies $s$ with redshifts of $z_{s}=z$ to get the mean surface density as a function of redshift $\Sigma_{P}(\theta,z)=\frac{1}{\mathcal{S}_{s}}\sum_{z_{s}=z}\Sigma_{sP}(\theta,z_{s})$.

\item Calculate the angular cross-correlation between $P$ and $S$ as $w_{SP}(\theta,z)=\frac{\Sigma_{P}(\theta,z)}{\bar{\Sigma}_{P}}-1$.  

\item Consider the analogous quantities in real-space: the mean number density $\bar{n}_{P}=\frac{N_{P}}{V_{P}(\{\hat{z}_{p}\})}$ of galaxies in $P$ in terms of the total number of galaxies $N_{P}$ in $P$ and the comoving volume $V_{P}(\{\vec{\theta}_{p}\}_{P},\{\hat{z}_{p}\}_{P},\{\vec{\theta}_{p}\}_{P})$ spanned by $P$, the number density $n_{sP}(r,z_{s})=\frac{N_{P_{s}}}{V(r,z_{s})}$ in terms of the number $N_{P_{s}}$ of galaxies in $P$ with $\hat{z}_{p}=z_{s}$ and the comoving volume $V(r,z_{s})$ within a distance $r$ of galaxy $s$ at redshift $z_{s}$, and the mean number density as a function of redshift $n_{sP}(r,z)=\frac{1}{\mathcal{S}_{s}}\sum_{z_{s}=z}n_{sP}(r,z_{s})$.

\item Note that the angular cross-correlation function $w_{sP}(\theta,z)$ is related to the real-space comoving cross-correlation function $\xi_{SP}(r,z)=\frac{n_{P}(r,z)}{\bar{n}_{P}}-1$.

\item Assume Limber's Equation of $\xi_{SP}(r)=(\frac{r}{r_{0}})^{-\gamma}$.

\item Recalling that the angular diameter distance is defined as $d_{A}(z)=\theta r(z)$, rewrite the angular cross-correlation function as $w_{SP}(\theta,z)=\frac{\phi_{P}(z)\mathcal{H}(\gamma)\theta d_{A}(z)}{(\frac{r}{r_{0}})^{-\gamma}\frac{dl}{dz}}$ in terms of the desired redshift probability distribution $\phi_{P}(z)$, the comoving distance $l(z)=c\int_{0}^{z}H^{-1}(z')dz'$ with the Hubble parameter $H(z)$, and the mathematical function $\mathcal{H}(\gamma)=\frac{\Gamma(\frac{1}{2})\Gamma(\frac{\gamma-1}{2})}{\Gamma(\frac{\gamma}{2})}$.

\item Solve for $\phi_{P}(z)$ in the above equation for $w_{sP}(\theta,z)$.

\item Profit?

%\item Calculate $\Sigma_{SP}(\theta,\Delta_{k})=\frac{1}{N_{S}(\Delta_{k})}\sum_{s_{k}=1}^{N_{S}(\Delta_{k})}\Sigma_{sP}(\theta,z_{s})$ as the average of $\Sigma_{sP}(\theta,z_{s})$ over all $N_{S}(\Delta_{k})$ objects $s_{k}$ in set $S$ within bin $k$.

%\item Define some set of physically motivated distances $r_{max}(z_{s})$ correponding to the maximum distance over which one would expect gravitational interactions between neighboring galaxies, and calculate the angular distance $\theta_{max}(z_{s})=\frac{r_{max}(z_{s})}{l(z_{s})}$ at a comoving distance $l(z_{s})=c\int_{0}^{z_{s}}H^{-1}(z')dz'$.
\end{enumerate}

This method has not yet been adapted to handle individual redshift posteriors for the test set and permits neither refinement of the input photometric redshifts nor determination of individual redshift posteriors based on $n(z)$ and $\vec{\xi}_{sp}$.  It is worth reviewing the method in detail to investigate possible extensions to zPDFs.
%\acknowledgments

%\appendix

\bibliography{references}

\end{document}