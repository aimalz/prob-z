%\documentclass[iop]{emulateapj}
%\documentclass[12pt, preprint]{emulateapj}
\documentclass[12pt, onecolumn]{emulateapj}

\usepackage{tikz}
\usetikzlibrary{shapes.geometric, arrows}

\newcommand{\myemail}{aimalz@nyu.edu}
\newcommand{\textul}{\underline}

\slugcomment{NOTE: THIS NEEDS SOME SERIOUS WORK WITH ORGANIZATION FOR CLARITY\dots}

\shorttitle{Probabilistic Redshift Notes}
\shortauthors{Malz}

\begin{document}

\title{Probabistic Redshifts}

\author{A.I. Malz\altaffilmark{1}}
\altaffiltext{1}{CCPP}
\email{aimalz@nyu.edu}

\begin{abstract}
Herein a methodology is presented for estimating physical (and in particular, cosmological) parameters using a probabilistic model of redshifts in place of point estimates.  
\end{abstract}

\keywords{cosmology; photo-z}

\section{Introduction}

(REVIEW LITERATURE FOR HOW WRONG POINT ESTIMATES CAN BE AND HOW GOOD THEY NEED TO GET FOR PRECISION COSMOLOGY.  ALSO INCLUDE CURRENT METHODS FOR OBTAINING REDSHIFT PROBABILITY DISTRIBUTIONS.)

This paper presents a technique for the correct use of probabilistic photometric redshifts in cosmological studies. 

\section{Method}

Consider a survey of $N$ galaxies conducted with a goal of determining cosmological parameters $\vec{\Omega}$ that comprise a model.  Let us establish that the probability that a galaxy $i\in[1,\dots,N]\in\mathbb{N}$ has cosmological redshift $z_{j}\in[0,\infty]\in\mathbb{R}$ is $p_{i}(z_{j})$ and assume that we are given all available $p_{i}(z_{j})$ as discrete probability distributions over a shared set of $z_{j}$.  Further, let us consider the redshift of a given object to be a random variable with some probability distribution $p(z)$.  
%REVIEW MATH?  The expectation of some function $f(z)$ of the redshift will be $\langle f(z)\rangle=\sum_{z}f(z)p(z)$ and its variance is $\langle(z-\langle z\rangle)^{2}\rangle=\langle z^{2}\rangle-\langle z\rangle^{2}=\sum_{z}z^{2}p(z)-(\sum_{z}zp(z))^{2}$.  

Bayes' theorem relates the conditional probability of the model given the redshift to the redshift given the model according to Eq. \ref{eq:bayes}.  In the absence of existing knowledge about cosmology, the intrinsic distribution $p(\vec{\Omega})$ of the parameters would be assumed to be uniform or perhaps data-based via the Jeffreys prior, however a realistic problem would use the results of other observational programs.  Given $p(z)$, we wish to find $p(\vec{\Omega},z)=p(\vec{\Omega}|z)p(z)$ or $p(\vec{\Omega})=\sum_{z}p(\vec{\Omega}|z)p(z)$.  In contrast to most maximum likelihood-based methods that simply find $p(\vec{\Omega}|z)$ assuming $p(z)$ is a delta function or occasionally a Gaussian centered at the true value, we wish to study the full likelihood function.

\begin{eqnarray}
\label{eq:bayes}
p(\vec{\Omega}|z) &=& \frac{p(z|\vec{\Omega})p(\vec{\Omega})}{p(z)}
\end{eqnarray}

Let us express the likelihood of an observation $\vec{d}_{i}$ consisting of a galaxy's measured apparent magnitude in several bands and associated errors.  This datum is an estimate of the true apparent magnitude of the object $m_{i}$ and its redshift $z_{i}$.  These are related to one another by way of the distance to the galaxy given in Eq. \ref{eq:D}, assuming a flat $\Lambda$CDM cosmology with known constants $c$, $H_{0}$, $\Omega_{M}$, and $\Omega_{\Lambda}$ in terms of its redshift.  Thus a cosmology defined by parameters in $\vec{\Omega}$ determines $z_{i}$ and $m_{i}$.   

\begin{eqnarray}
\label{eq:D}
D(z) &= \frac{c}{H_{0}}\int_{0}^{z}\frac{dz'}{\sqrt{\Omega_{M}(1+z)^{3}+\Omega_{\Lambda}}}
\end{eqnarray}

The apparent magnitude depends not only on this distance and thus the cosmological parameters $\vec{\Omega}$ but also on the intrinsic brightness of the galaxy $M_{i}$ according to Eq. \ref{eq:M}.   (K-CORRECTIONS WILL BE IGNORED FOR NOW BECAUSE I'M NOT SURE HOW TO INFER A CONTINUOUS LUMINOSITY DISTRIBUTION FROM FIVE BANDS OF PHOTOMETRY, I.E. HOW DO I GET $L_{\lambda/(1+z)}$ FROM $L_{R}$ FOR BANDPASS R AS IN \citet{hog99}?)

\begin{eqnarray}
\label{eq:M}
M(D(z)) &= m-5\log(D/10pc)
\end{eqnarray}

The absolute magnitude of a galaxy is drawn from a distribution determined by galaxy diversity parameters $\vec{\Phi}$.  Nuisance parameters $\psi$ will be introduced by the method used to determine its redshift, such as the generation of galaxy SED templates, but those parameters ought to be closely related to $\vec{\Phi}$.  (I'M STILL NOT 100\% CLEAR ON HOW THESE INFLUENCE THE OBSERVATION THOUGH I CAN SEE HOW THEY FACTOR INTO OUR ABILITY TO ESTIMATE $z$ and $m$.  I MIGHT BE GETTING STUCK ON SOMETHING RELATED TO THE NOTATION.  $\psi$ MAY OR MAY NOT DIFFER PER GALAXY, CORRECT?)  These hierarchical relationships are illustrated in Fig. \ref{fig:flow}.  

%Other relevant physics related to galaxy evolution may be rolled into another set of parameters $\vec{\Theta}$ that influence $\vec{\Phi}$.  

\begin{figure}
\vspace{0.5cm}
\begin{center}
\label{fig:flow}
\tikzstyle{hyper} = [rectangle, rounded corners, minimum width=1cm, minimum height=0.5cm,text centered, draw=black, fill=yellow!30]
\tikzstyle{param} = [trapezium, trapezium left angle=70, trapezium right angle=110, minimum width=1cm, minimum height=0.5cm, text centered, draw=black, fill=green!30]
\tikzstyle{obs} = [rectangle, minimum width=1cm, minimum height=0.5cm, text centered, draw=black, fill=blue!30]
\tikzstyle{data} = [diamond, minimum width=1cm, minimum height=1cm, text centered, draw=black, fill=red!30]
\tikzstyle{arrow} = [thick,->,>=stealth]

\usetikzlibrary{fit}
\begin{tikzpicture}[node distance=1cm]
\label{}
%\node (phys) [hyper, xshift=-1cm] {$\vec{\Theta}$};
\node (cosmo) [hyper, xshift=+1cm] {$\vec{\Omega}$};
\node (LF) [hyper, xshift=-1cm] {$\vec{\Phi}$};
\node (lum) [param, below of=LF] {$M_{i}$};
\node (mod) [obs, below of=lum, xshift=-2cm] {$\psi$};
\node (mag) [obs, below of=lum] {$m_{i}$};
\node (red) [obs, below of=lum, xshift=+2cm] {$z_{i}$};
\node (dat) [data, below of=mag] {$\vec{d}_{i}$};
\node (gal) [draw=gray,fit={(mag.west)([yshift=0.2cm] lum.north)(dat.south)(red.east)}] {};
\node [anchor=north] at (gal.north) {Observed Galaxies};

%\draw [arrow] (phys) -- (LF);
\draw [arrow] (cosmo) -- (mag);
\draw [arrow] (cosmo) -- (red);
\draw [arrow] (LF) -- (lum);
\draw [arrow] (LF) -- (mod);
\draw [arrow] (lum) -- (mag);
\draw [arrow] (red) -- (mag);
\draw [arrow] (mag) -- (dat);
\draw [arrow] (mod) -- (dat);
\draw [arrow] (red) -- (dat);

\end{tikzpicture}
\caption{This directed acyclic graph illustrates the hierarchical model one can use for the likelihood in this problem.  The observed data $\vec{d}_{i}$ is influenced by the true redshift $z_{i}$, apparent magnitude $m_{i}$, and the parameters $\vec{\psi}_{i}$ used to fit those physical properties.  By Eqs. \ref{eq:D} and \ref{eq:M}, the cosmology, redshift, and absolute magnitude influence the apparent magnitude.  The absolute magnitude and fitting parameters are influenced by the luminosity function parametrized by $\vec{\Phi}$%, which is in turn influenced by the physical parameters governing galaxy evolution herein parametrized as $\vec{\Theta}$
. The cosmological parameters of $\vec{\Omega}$ influence both the redshift and apparent magnitude of each observed galaxy.}
\end{center}
\end{figure}

Thus the likelihood we seek is Eq. \ref{eq:likelihood}.  (I THINK THIS IS CONSISTENT WITH THE FLOWCHART, BUT I'M NOT SURE IT'S COMPLETE NOR CORRECT, NOT TO MENTION THAT THE LAST TERM MAY BE CONSIDERED UNKNOWABLE.)


\begin{eqnarray}
\label{eq:likelihood}
p(\vec{d}_{i}|\vec{\Phi},\vec{\Omega}) &=& p(\vec{d}_{i}|z_{i},m_{i},\psi_{i})p(\psi_{i}|\vec{\Phi})p(m_{i}|z_{i},M_{i},\vec{\Omega})p(M_{i}|\vec{\Phi})p(z_{i}|\vec{\Omega})
\end{eqnarray}

Now let us consider the probability of observing an ensemble of data $\textbf{d}=(\vec{d}_{1},\dots,\vec{d}_{N})$.  If the data were uncorrelated, we would have Eq. \ref{eq:independent}.  However, individual galaxies with observations $\vec{d}_{i}$ will be correlated by way of $\Phi$ and $\vec{\Omega}$ so we seek Eq. \ref{eq:dependent} assuming an ordering on each new data point $\vec{d}_{i}$ observed.  (THIS ISN'T REALLY WHAT I WANT, BUT IT'S WHAT I MOST EASILY KNOW HOW TO WRITE DOWN.  \citet{for14} SHOWS THAT THIS CORRELATION DOES NOT VIOLATE STATISTICAL INDEPENDENCE SO I THINK I'LL DO THE SAME.)

\begin{eqnarray}
\label{eq:independent}
p(\textbf{d}|\vec{\Phi},\vec{\Omega}) &=& \prod_{i}p(\vec{d}_{i}|\vec{\Phi},\vec{\Omega})
\end{eqnarray}

\begin{eqnarray}
\label{eq:dependent}
p(\textbf{d}|\vec{\Phi},\vec{\Omega}) &=& p(\vec{d}_{1}|\vec{\Phi},\vec{\Omega})p(\vec{d}_{2}|d_{1},\vec{\Phi},\vec{\Omega})\dots p(\vec{d}_{N}|\vec{d}_{1},\dots,\vec{d}_{N-1},\vec{\Phi},\vec{\Omega})
\end{eqnarray}

\section{Application}

The application of probabilistic redshifts is herein demonstrated on the luminosity function, a one-point statistic in the form of a density in the intrinsic brightness of galaxies.  

% \begin{mathletters}
% \begin{eqnarray}
% \label{eq:MK}
% M_{\lambda}(D(z)) &= m_{\lambda}-5\log(D/10pc)-2.5\log(\frac{1}{1+z}\frac{L_{\lambda/(1+z)}}{L_{\lambda}})
% \end{eqnarray}
% \end{mathletters}

The luminosity function $\Phi(M)$ is the number density of galaxies as a function of absolute magnitude and is typically parametrized according to the Schechter Function of Eq. \ref{eq:schechter} in terms of the parameters $\phi_{*},M_{*},\alpha$.  Non-parametric estimators are available, such as the stepwise maximum likelihood method of \citet{efs88}.  Such estimators account for the fact that galaxy samples are typically truncated in apparent magnitude due to the observational limits, and these translate to nonuniform truncation in absolute magnitude since galaxies are not all located at the same redshift.  

\begin{eqnarray}
\label{eq:schechter}
\Phi(M) &=& \frac{\ln10}{2.5} \phi_{*}(10^{-(M-M_{*})/2.5})^{\alpha+1}\exp[-10^{-(M-M_{*})/2.5}]
\end{eqnarray}

Usually, the errors reported with the parameters $\phi_{*},M_{*},\alpha$ are statistical errors related to sample size and variance from the fit, not proprgations of the observational uncertainties.  When observational uncertainties are included, they are generally assumed to be Gaussian and provided as $1\sigma$ limits $(z_{-},z_{+})$ and $(m_{-},m_{+})$ around point estimates $z_{0}$ and $m_{0}$.  

\citet{bla02} incorporates appropriate observational uncertainties on apparent magnitude.  Because observations of $m$ are in fact Gaussian distributions $p(m)\sim N(m_{0},\sigma_{m}^{2})$ around the true apparent magnitude, they may be propagated to Gaussian probability distributions on $p(M)\sim N(M,\sigma_{M}^{2})$, which in turn may be passed on to the non-parametric form of the discrete luminosity function of Eq. \ref{eq:blanton} (leaving out terms permitting evolution of absolute magnitude and number density with redshift). 

\begin{eqnarray}
\label{eq:blanton}
\Phi(M) &=& \bar{n}\sum_{k}\frac{\Phi_{k}}{\sqrt{2\pi\sigma_{M}^{2}}}\exp[-(M-M_{k})^{2}/2]
\end{eqnarray}

Knowing $p(z)$, we may find the probabilistic distance distribution $p(D)$ by changing variables according \ref{eq:D}, assigning probability $p(z')$ to each $D'(z')$ to get $p(D')$.  Similarly, we can find the probabilistic brightness $p(M)$ according to \ref{eq:M}, assigning a probability $p(z')$ to each $M'(D'(z'),m)$.  ($\Delta m$ IS IGNORED HERE AND SHOULD BE INCLUDED FOR REASONABLE ERROR ANALYSIS -- THIS IS THE NEXT THING I'M WORKING ON.)

Here we estimate the luminosity function of an ensemble of galaxies each with a given $p_{i}(z_{j})=p(d_{i}|z)\approx p_{i}(\vec{d}|z,m,\vec{\psi})$ distributions.  In this work, evolution of luminosity with redshift is neglected.  The number of galaxies observed as a function of absolute magnitude is given in Eq. \ref{eq:N}, recalling the absolute magnitude of Eq. \ref{eq:M}.  To get a number density of galaxies by absolute magnitude over this redshift range, we would use Eq. \ref{eq:n}, recalling the distance $D$ from Eq. \ref{eq:D}.  Since the survey is magnitude limited at $m_{lim}$, there will be an associated $M_{lim}(z_{j})$ for each redshift.  (SOMEHOW I NEED TO ACCOUNT FOR TRUNCATION LIMITS HERE\dots)

\begin{eqnarray}
\label{eq:N}
N(M) &= \sum_{i}\sum_{j}p_{i}(M|p_{i}(z_{j}),m_{i})
\end{eqnarray}

\begin{eqnarray}
\label{eq:n}
n(M) &= \sum_{i}\sum_{j}\frac{p_{i}(M|p_{i}(z_{j}),m_{i})}{\frac{4}{3}\pi D^{3}(z_{j})}
\end{eqnarray}

\section{Results}

Preliminary data exploration began with examination of a randomly chosen set of 155,262 redshift probability distributions.  The galaxies were pseudo-randomly chosen from SDSS DR8 based on their accessibility at \url{http://data.sdss3.org/sas/\\dr8/groups/boss/photoObj/photoz-weight/}.  These were calculated using a $k$-nearest neighbors algorithm presented by \citet{she11}.  (GO OVER DETAILS TO DOUBLE-CHECK IF THIS IS ACTUALLY $p(d_{i}|z_{i})$.)  

3185 galaxies also had spectroscopic redshifts available.  In most cases, the photometric redshift was closer to the true redshift than the expected value of the probabilistic redshift, although this is not surprising because the expected value loses much of the valuable information contained in the distribution.  The pseudo-luminosity function of Eq. \ref{eq:N} was calculated with each of these two point estimators of redshift and the probabilistic redshift, as well as the true redshift and is shown in Fig. \ref{fig:lfs-spec}.  (I WILL ATTEMPT TO FIT A SCHECHTER FUNCTION TO THESE JUST TO SHOW HOW DIFFERENT THE RESULTS ARE.)  The actual number density as a function of absolute magnitude of Eq. \ref{eq:n} is shown in Fig. \ref{fig:lfs-spec-vol}.

\begin{figure}
\plotone{lf-z-compare-big.png}
\caption{On the whole, the absolute magnitude number density is unaffected by the inaccuracy of photometric redshifts.  The probability distributions also produce a number density consistent with the spectroscopically-confirmed one.  Unsurprisingly, the expected value does not accurately reproduce the shape of the curve and gives a conflicting result.  \label{fig:lfs-spec}}
\end{figure}

\begin{figure}
\plotone{lf-z-compare-vol.png}
\caption{One can see that the spectroscopic redshifts lead to a different luminosity function at the dim end, while probabilistic redshifts extend the luminosity function to a wider range of brightnesses.  Since all these curves represent a single set of galaxies, there is some indication that photometric and probabilistic methods are incorrectly assigning higher redshifts to nearby objects.   \label{fig:lfs-spec-vol}}
\end{figure}

However, the galaxies with spectroscopic redshifts are a highly biased subsample of the total number of galaxies observed, representing only the brightest galaxies.  Photometric redshifts are at their worst for dimmer galaxies.  (CITE THIS!)  The redshift probability distributions for galaxies whose redshifts could be confirmed spectroscopically are much more strongly peaked and at lower redshifts than those that were not confirmed spectroscopically, as seen in Fig. \ref{fig:zdists}.

\begin{figure}
\plotone{zdist.png}
\caption{It can be seen that the probability distributions of redshifts in in the spectroscopic sample differs greatly from the probability distributions of redshifts in the sample not observed spectroscopically.  \label{fig:zdists}}
\end{figure}

\section{Discussion}

PROBABLY SHOULD INCLUDE SOME ACTUAL DISCUSSION\dots

Of more significance would be consideration of a two-point statistic of the redshift probability distribution such as the power spectrum $mathcal{P}(r,z)$.  We would need the distances $r=|\vec{r}|=|\vec{x}-\vec{x}'|$ between galaxies, and thus the probabilities that those galaxies are at each location in space $p_{i}(\vec{x})$.  

IT'S ALL SPECULATION FROM HERE ON OUT.  For a single pair of galaxies we have Eq. \ref{eq:separation}.  (I AM ASSUMING THE POWER SPECTRUM WOULD BE CALCULATED AS A FUNCTION OF REDSHIFT SUCH THAT ONLY GALAXIES IN THE SAME REDSHIFT BIN WOULD BE CONSIDERED.)  Eq. \ref{eq:power} gives the power spectrum $P(r,z_{k})$ for the ensemble at each redshift.  (THIS NEEDS TO BE NORMALIZED.)  Since the power spectrum has well-measured properties, the posterior $p(z,r|\mathcal{P}(r,z))$ is known.  

\begin{eqnarray}
\label{eq:separation}
p_{ij}(r,z_{k}) &=& p_{i}(z_{k},\alpha_{i},\delta_{i})p_{j}(z_{k'},\alpha_{j},\delta_{j})
\end{eqnarray}

\begin{eqnarray}
\label{eq:power}
P(r,z_{k}) &=& \sum_{i\neq j}p_{ij}(r,z_{k})
\end{eqnarray}

The result of such an analysis could in turn be used to update $p_{i}(z_{k})$ distributions to more likely values.  This would be especially of interest in light of Fig. \ref{fig:zdists}, which shows that samples for which spectroscopic redshifts are available may have little in common with samples for which spectroscopic redshifts cannot be obtained, so any method that is based on well-measured spectra is limited to well-behaved objects.  Bayesian updating of redshift probability distributions via well-known cosmology results could help with galaxy evolution problems, such as redshift evolution in the luminosity function.

%\acknowledgments

% \appendix

\begin{thebibliography}{}
\bibitem[Sheldon, et al. (2011)]{she11}
Sheldon, E.S., Cunha, C., Mandelbaum, R., Brinkmann, J., Weaver, B.A., arxiv:1109.5192
\bibitem[Blanton, et al. (2002)]{bla02}
Blanton, M.R., et al., arxiv:astro-ph/0210215
\bibitem[Efstathiou, et al. (1988)]{efs88}
Efstathiou, G., Ellis, R.S., Peterson, B.A.
\bibitem[Foreman-Mackey, Hogg, and Morton (2014)]{for14}
Foreman-Mackey, D., Hogg, D.W., and Morton, T.D., arxiv:1406.3020
\bibitem[Hogg (1999)]{hog99}
Hogg, D.W., arxiv:astro-ph/9905116
\bibitem[Hogg (2012)]{hog12}
Hogg, D.W., arxiv:1205.4446

FILL IN MORE OF THESE!
\end{thebibliography}

\end{document}